\setchapterpreamble[u]{\margintoc}
\chapter{文档类型选项}
\labch{文档类型选项}

本章将介绍最常用的选项,这些选项来自 \Class{scrbook} 以及 \Class{kaoCJKsc} 特有的部分。给模板设置选项将会改变其默认的行为,但要小心,有些选项会造成意想不到的结果……

\section{\Class{KOMA} 选项}

\Class{kaobookCJKsc} 模板基于 \Class{scrbook},因此,继承而来的设置选项同样可以生效。如果你有耐心,可以读一读 \KOMAScript 的使用手册。\sidenote{手册可以从这里下载 \url{https://ctan.org/pkg/koma-script?lang=en}。}实际上,建议读一下类似的使用手册,因为这些包含很多指导性的内容。

设置给模板的每一个 \KOMAScript 选项,会被自动加载。另外,在 \Class{kaobookCJKsc} 模板中,已经修改了部分默认的设置。比如,字体大小默认设置为 9.5pt,段落之间会使用空格分割\sidenote[][-7mm]{强调一下,段落间距默认设置为一半的行距:\Option{parskip} 值为 \enquote{half}。} ,并且会缩进 2 个汉字。

\section{\Class{kaoCJKsc} 选项}

未来计划加入一些设置段落格式的选项,以及边注位置的选项。\marginnote{译者注:原模板的段落是每行对齐的,\Class{kaobookCJKsc} 模板已经按中文习惯设置首行缩进。}借此机会,呼吁大家踊跃添加特性,重新设计现有特性,并能将结果发给我。可以在这里查看原版仓库 \url{https://github.com/fmarotta/kaobook}。

\begin{kaobox}[frametitle=\noindent 待办]
Implement the \Option{justified} and \Option{margin} options. To be consistent with the \KOMAScript style, they should accept a simple switch as a parameter, where the simple switch should be \Option{true} or \Option{false}, or one of the other standard values for simple switches supported by \KOMAScript. See the \KOMAScript documentation for further information.
\end{kaobox}

以上是使用 \Environment{kaobox} 的示例,将会在\nrefch{数学}详细介绍。本书会使用这样的块,标记将要实现的一些特性。

\section{其他值得关注的东西}

因实现的需要,一大堆包已经加载在该模板里了:

\begin{itemize}
	\item etoolbox
	\item calc
	\item xifthen
	\item xkeyval
	\item xparse
	\item xstring
\end{itemize}

对于大多数包,会在合适的时间去讨论。此刻,只提及一些用于段落格式调整的包(别忘了,未来这些可能被调整),主要包括以下这些:

\begin{itemize}
	\item ragged2e
	\item setspace
	\item hyphenat
	\item microtype
	\item needspace
	\item xspace
	\item xcolor (with options \Option{usenames,dvipsnames})
\end{itemize}

以上这些,有的不是针对段落格式的,但我们还是归到一起了。默认情况下,原模板主体文本和边注都是每行对齐的,而中文版主体文本调整为首行缩进,边注保持原样。

最后提醒一下,\Package{cleveref} 包与 \Class{kaobookCJKsc} 模板不兼容,请使用\nrefsec{hyprefs}讨论的命令替代。

\section{文档结构}

我们给 \Command{title} 和 \Command{author} 命令提供了可选参数,可以在前置文档中插入简短的文字。相应的,也提供了 \Command{@plaintitle} 和 \Command{@plainauthor} 命令,用于插入无样式的文字。pdf 属性由 \Option{pdftitle} 和 \Option{pdfauthor} 两个选项自动设置,也可以通过 hyperref 超链接设置具体值。\sidenote[][*-1]{由于这很重要,在此强调一下。如果 pdf 元数据被修改,编译时会设置为相应值;否则,元数据默认为文档的标题和作者。}

定义了两种页面排版 \Option{margin} 和 \Option{wide},以及两个页面样式 \Option{plain} 和 \Option{fancy}。排版只关注页边距,而样式指的是页眉和页脚。这些会在\nrefch{排版}讨论。\sidenote{目前只需要了解,\Option{margin} 排版有较宽的页边距,而 \Option{wide} 排版没有页边距。\Option{plain} 样式没有页眉页脚,而 \Option{fancy} 样式有页眉但是没有页脚。}

\Command{frontmatter},\Command{mainmatter} 和 \Command{backmatter} 三个命令重新定义以便自动切换页面的排版和样式。前置文档使用 \Option{wide} 排版和 \Option{plain} 样式。正文部分采用 \Option{margin} 排版和 \Option{fancy} 样式。附录部分保持不变;不过,我们可以使用 \Command{bookmarksetup\{startatroot\}} 命令,把章节书签等级设置为 root 级别,否则将会追加在前面的章节里。后置文档重新使用 \Option{wide} 排版,并重设了书签等级。
